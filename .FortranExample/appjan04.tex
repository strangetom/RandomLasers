
ABSTRACT:


\section{Introduction}

The laser is now a familiar and well-understood system in physics, with many
applications in industry, medicine and research. Almost all
present-day lasers,
however, rely for their operation on a tuned optical cavity, formed by
mirrors, which largely controls the frequency and spectral purity of the
output laser light. Without an optical cavity of this type, there is usually
insufficient amplification of light in the laser medium to place the laser
light above threshold intensity.

The present application seeks funding to study an alternative form of laser:
the so-called random, or natural, laser mechanism. In random lasing, there
is still a laser medium, containing the active atoms or molecules, but
there is no mirrored cavity to control the light output. Instead, sufficient
amplification of the laser light is generated by the formation of natural
cavities within the medium. As in most ordinary lasers, the scale-size of
the laser medium is too small to give rise to a single-pass laser, so the
amplification process requires a process, other than specular reflection,
which can change the direction of light propagation within the lasing
medium. The usual process invoked to construct natural cavities is repeated
large-angle scattering. The scattering centres may be the lasing atoms or
molecules themselves, or other components of the lasing medium. The name
`random laser' arises from the formation and dissolution of the natural
cavities: closed paths for amplification, resulting from repeated scatterings,
are subject to changes within the laser medium, including the intensity of
the laser-light itself. As amplification paths develop, both the intensity
and direction of the laser output will change in a manner which appears
to be random.

Whilst today's man-made lasers are cavity controlled, natural counterparts
exist in astrophysics. These are cosmic, or astrophysical, masers which can
be formed in a variety of environments, from the comets in our own Solar
System, to star-forming regions, circumstellar envelopes and the cores of
external galaxies. Most of these astrophysical masers have amplification
factors which can be understood on the basis of a single pass of the
radiation along a gain-length with only a small angular spread (of order
10$^{\rm -2}$\,radians. However, in the last case mentioned, the masers in
the cores of external galaxies, there are bright hot spots which are
difficult to explain on the basis of a single-pass gain length. These sources
are known as `megamasers', because their typical power output is rougly a
million times greater than that of a maser in a Galactic star-forming region.
The typical structure of a megamaser consists of a low intensity diffuse
component, emitted by a broad region, perhaps 0.1\,kiloparsecs in extent,
and a group of hot-spots, only apparent when
observed at very high angular resolution. The diffuse component can be
explained acceptably by a single-pass amplification, but to generate the
hot-spots may require a more complex path. One possibility is the formation
of natural cavities, much as in the laboratory random lasers, but on a
vastly larger scale. Normally, interstellar scattering produces angular
changes in the direction of propagation of radio waves which are far too
small to form loops, or other closed figures, on a reasonable scale, but in
megamaser sources we find both large distances, and large refractive indicies,
based on high degrees of ionization.

In addition to being a fascinating natural process, and the rather esoteric
application to astrophysics, on a scale at least $10^{\rm 20}$ times larger
than typical laboratory samples, we suggest that random lasing may have
important industrial applications in the near future. One of these is to
produce laser light of modest quality (such as is currently generated by
semiconductor lasers) in situations where electronic circuits have been
constructed without the shaping provided by current manufacturing
techniques. At present, much research is in progress towards constructing
circuits by a process more akin to ink-jet printing of components onto a
flexible substrate (Sirringhaus et al. ??). It would be very difficult to
simulate a traditional semiconductor laser via this process, but a random
laser, without internal structure, would be ideal. There remains, of course,
the problem of ensuring that the output light is at least somewhat directed.
Other applications could take advantage of the very rapid natural switching
of the random lasers: re-arrangement of natural cavity structures can lead
to fluctuations in intensity of more than an order of magnitude over 
timescales of a nanosecond or less.

Since the amplification process is highly non-linear, it may be possible to
use the fast switching properties of the random laser to amplify signals
at frequencies other than the laser line.

\section{Key Questions}

(1) Can we understand the process of random lasing in as simple a manner as
possible, so that we can predict and control the paramters of the output
light, such as its intensity, phase properties, fluctuation timescales
and direction.

(2) Is it possible to explain megamaser hotspots in astrophyics in terms of
the formation of natural laser cavities of galactic scale by interstellar
scattering?

(3) Is it possible to produce random lasing in plastic materials which can
be applied to the proposed `printing' method of manufacturing electronic
(and/or photonic) circuits.


\section{Scientific Case}

The theory of random lasing goes back to work by Letokhov (1968), who showed
that there is an analogy between the amplification of the population of
photons in a stongly scattering laser medium and the amplification of the
population of neutrons in a chain reaction in a fissile material, such
as $^{\rm 235}$U. At this stage, the amplifying medium was assumed to be
disordered, leading to incoherent laser emission.

Random lasing relies on two processes: one is the standard amplification by
stimulated emission found in the ordinary laser; the other is the formation
of natural optical traps or cavities, which can store light in the medium
for times large compared with the light-crossing time of the sample. This
second process was first observed experimentally by Garcia \&
Genack (1991) and Genack \& Garcia (1991). These experiments were carried out
with microwaves and demonstrated weak localization (temporary trapping) of
the radiation. We note that this radiation can later be released from the
material, differentiating the trapping process from absorption, in which the
energy of the photons is dissipated. It has been shown experimentally that
photon localization does occur (Schuurmans et al. 1999), and that absorption cannot explain the
results observed, regarding the amplitude of fluctuations in the output
(transmitted) light (Chabanov et al. 2000; Chabanov \& Genack 2001a,b). More recently, experiments have produced photon
localization
at optical frequencies in semiconductor powders (Wiersma et al. 1997, 1999).

The combination of photon localization and an active lasing medium, leading to
random lasing was first observed by Lawandy et al. (1994), and it has 
subsequently been observed for several systems which differ in the type of
gain medium (for example semiconductor powders and colloidal suspensions,
including dyes). 
The temporary trapping of photons means that they no longer follow straight
paths through the sample. Instead they follow a path punctuated by many
scatterings which change the direction and phase of the light. This type
of motion has much in common with a random-walk process, and suggests that
the passage of weakly-localized light can be understood in terms of
diffusion theory. Diffusion theory has been shown to give a good description
of random laser systems (for example Wiersma \& Lagendijk 1996) provided
that coherence information in the laser light is not important. The authors
have carried out an initial computational
study of random lasing, using the diffusion
model, in 2-D and 3-D configurations, including a slab with a central source:
a crude analogue of the structure of a possible astrophysical system
(French 2003). Results in Wiersma \& Lagendijk (1996) were also verified, using
a computer code based on a different algorithm than the one used in that work.

So far we have assumed, through the use of the diffusion approximation, that
the output of a random laser is amplified, perhaps very strongly, but is
not coherent. Very recently, however, it has been shown through photon
statistics experiments (Cao et al. 2001; Polson et al. 2001) that when there is very strong scattering (strong
photon localization) random lasers exhibit coherence properties similar to
those of a traditional laboratory laser. This coherent random lasing is
explained in detail in Apalkov et al. (2003). Their work considers light
trapping in passive media, but their conclusions are almost certainly
extendable to active (laser) media (Cao et al. 2002).
Understanding the coherence
properties of the light requires a wave treatment, rather than the simpler
diffusion approximation. Apalkov et al. (2003) point out that there is a
mathematical
analogy between the Schr\"{o}dinger equation for an electron bound in a
random potential, and the wave equation for light propagating through a medium
which has a fluctuating refractive index. Solutions of the wave equation
(Apalkov et al. 2002)
yield `almost localized' states, in which light is trapped in small rings
of scattering centres, which play the role of natural high-Q resonators.
Compared to the incoherent variety,
coherent random lasing requires a smaller transport mean-free path for the
light: incoherent lasing is possible when the transport
mean-free path of the light becomes
significantly smaller than the sample size, but for coherent random lasing
this mean-free path needs to be comparable to the wavelength of the light.

Much of the physics of random lasing is still unknown, and we intend to
investigate several parameters of random laser systems under the terms of
the this proposal. The first, and perhaps the most crucial, parameter is
the number density of resonators: a property of the material, and perhaps
also of the light intensity in an active medium. The density of resonators
is expected to be a function of both the scattering strength of the medium,
and its level of microscopic disorder (Apalkov 2003). We intend to discover
the optimum material for random lasing, that is the material which will
produce the highest laser output per unit volume of sample. In looking for
practical devices, there are, of course, also considerations which might
rule out some theoretically promising materials, such as their toxicity and
cost of manufacture. The second regime of study is the effect of the 
geometry of the sample on the laser output. Much current study has been for
restricted geometries, usually for thin films, effectively of two dimensions.
We intend to carry out numerical computations of laser output in 3-D, with
emphasis on spheres, filaments and slabs of material, the latter being the
easiest structure to make for practical devices. The possible astrophysical
application is also part of the geometrical study, such a natural maser
being modelled as a disc with a central pumping source. We note that in
this case the maser would not have strong coherence, since a mean free path
of the order of the wavelength (\sim 18\,cm) is impossible to achieve for
such a system. We note, however, that all astrophysical masers are expected
to display some residual coherence (Gray \& Bewley 2003)
If time permits, we would also like to look at directability
of the laser output by the application of external perturbations, for 
example, the influence of strongly directed pumping light.

\section{Work Plan}

(1) For the first six months we intend to write a computer code to study both
incoherent and coherent random lasing. This will require the use of wave
theory, rather than the diffusion-based code available to the authors at
present. 

(2) For the next eighteen months, we intend to use the computer code to derive
the optimum properties for random lasing, and the effects of the sample
geometry on the output.

(3) A considerable amount of theoretical analysis will
be carried out concurrently with (1) and (2), as this theory guides the
input data for the computations.

(4) Six months will be devoted to the vastly scaled-up astrophysical problem:
 we intend to determine whether incoherent random lasing is an explanation
for megamaser hotspots.

(5) Any remaining time, up to six months, will be used to investigate the
effect of external perturbations of the propeties of the random laser
systems. The principal effect to be studied will be the directionality of
the pumping light.

\section{Resources Requested}

PDRA, 3 yrs +
(A big computer, usual travel, computer officer support and secretarial
support at standard rates??)

\section{References}

Apalkov V.M., Raikh M.E., Shapiro B., 2002, Phys. Rev. Lett., {\bf 89}, 016802
Apalkov V.M., Raikh M.E., Shapiro B., 2003, \\
Cao H., Ling Y., Xu J.Y., Cao C.Q., Kumar P., 2001, Phys. Rev. Lett. {\bf 86},
4524\\
Cao H., Ling Y., Xu J.Y., Burin A.L., 2002, Phys. Rev. E, {\bf 66}, 025601(R)
Chabanov A.A., Stoytchev M., Genack A.Z., 2000, Nature, {\bf 404}, 850\\
Chabanov A.A., Genack A.Z., 2001, Phys. Rev. Lett., {\bf 87}, 153901\\
Chabanov A.A., Genack A.Z., 2001, Phys. Rev. Lett., {\bf 87}, 233903\\
French S.T., 2003, `A Computational Model of Random Lasing', Dept. of Physics,
UMIST, UK\\
Garcia N., Genack A.Z., 1991, Phys. Rev. Lett., {\bf 66}, 1850\\
Genack A.Z., Garcia N., 1991, Phys. Rev. Lett., {\bf 66}, 2064\\
Gray M.D., Bewley S.L., 2003, MNRAS, {\bf 344}, 439
Lawandy N.M., Balachandran R.M., Gomes A.S.L., Sauvain E., 1994, Nature, {\bf 368}, 436\\
Letokhov V.S., 1968, Soviet Physics JETP {\bf 26}, 835\\
Polson R.C., Chipouline A., Vardeny Z.V., 2001, Adv. Materials, {\bf 13}, 760
Schuurmans F.J.P., Megens M., Vanmaekelbergh D., Lagendijk A., 1999, Phys. Rev.
Lett., {\bf 83}, 2183\\
Wiersma D.S., Bartolini P., Lagendijk A., Righini R., 1997, Nature, {\bf 390},
671\\
Wiersma D.S., Rivas J.G., Bartolini P., Lagendijk A., Righini R., 1999, Nature, {\bf 398}, 207\\
Wiersma D.S., Lagendijk A., 1996, Phys. Rev. E, {\bf 54}, 4256\\

